前 言

中国菜烹制艺术異常丰富,且有悠久的历史,在国內外均 享有很高声誉。作为我国宝貴的文化遺产,烹飪技艺在解放后 受到党和政府的重視和关怀,得
到了空前的發展和提高。社会 主义建設事業的飞躍發展和人民羣众物質文化生活水平的迅速 提高,向烹飪工作提出了新的任务,即必須为生产服务,在
飮 食方面最大限度地滿足人民要求,使人民吃得好而富于衛生营 养。

1958年偉大的整風运动和社会主义大躍进,使北京飯店全 体厨师政治覚,圄有了显著提高,在不断挖掘傳統名菜的基础上 还积極創造新潁花样,增加和
丰富了菜的品种,經常地滿足着 顧客的要求。由于克服了保守思想和在技术上刻苦鑽硏,作出 的菜已更加合乎色香味美,在成菜裝飾的艺术性和餐具器
皿的 选配上也取得了相当的成效。

为了进一步总結我国烹飪技术經驗,在飯店党委和行政的 領导下,全体厨师解放崽想,于定質定量的基础上加以整理提 高,写成“北京飯店名菜譜”。

这本書的編写过程:是根据川菜名师范俊康、罗国荣;粵 菜厨师張桥、康輝;北方菜厨师王蘭、施文才、于業誠;譚家 菜厨师彭長海和点心师郭文彬等
九同志的口述記录,由李長 峰同志执笔編写而成的。在定稿过程中,我們召集了各厨师与 有关業务人員进行細致的討論,並根据大家所提意見作了若干 必要的修正。

本書包括川菜143种,粵菜102种,北方菜52种,譚家 菜30种,点心50种,共計387种。各菜均詳細裁述用料数 量,質量、加工过程、烹制方法、操作程序、成菜裝飾、風味

特点和規格要求。此外:我們还插入烹調要略一編,作为調味 的理論指导,並于書后附选部分宴会菜單,以供选配菜餚参考 之用。

中国菜餚之丰,声冠世界。但我国傳統都是分师授徒,因 此在操作用料諸方面,虽为同一方菜,往往彼此差異甚大。甚 而一师一法,故不宜强求一致。

本書所列各菜用料系以一桌(10人)計算,大批加工用料 較省,少量制作用料較費。根据加工規模大小,数量多寡有極 大的伸縮性,下料时切不可死板
地按照作料表进行。唯求斟酌 情况,有所增減。

各地、各公共飮食企業設备条件不全然相同,企業加工和 家庭制作亦有所区別。爐灶形式、燃料种类、火力强弱、时間 迟速等因素,皆須考虑在內。进
行操作时,务要注意火候。本 書对燒炒时間作了估計,但恐不够准确,亦希根据具体情况灵 活掌握。

“北京飯店名菜譜‘‘是一个初产物,我們尙缺乏足够的經 驗,不可避免地有許多錯誤和不妥之处,希望各界同志予以批 評指正,以期將来繼續修正和提高。
