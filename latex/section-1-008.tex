8-点綴佈置

在色、香、味三个烹調要素之中,色为第一位。外表美規、 色澤分明的食物,能够給人一种愉快的感覚,引起胃液分泌和 增进食慾;外表不美覗,色澤混
烏的食品使人見之首先發生一 种厭惡情緒和反感,不吃即飽,那怕質量再好,味道再美也不 会为人喜爱和欢迎,因此菜的点綴和佈置就火冇了極为重要
的 意义。佈置时可利用胡兹卜、紅菜头刻成的雕花、紙花、荷 花、菊花及西紅柿、黄瓜、生菜叶等。菜的佈置要求带有艺术 性,要求發揮厨师高度的阮心,經常的硏究和揣摩、不断地創


造和改进。愈显生动活潑則愈有風味。进行佈置时,必須考虑 到色澤鮮明协調,配菜与主菜是否相称。不要使配菜遮盖主 菜,或因配菜多主菜少而造成
宣宾夺主。配菜一般应摆于盤的 周圍,掛汁的菜要注意不把汁澆在配菜上。
