7.成袭溫度

菜的温度或冷或热对菜的風味和質量好坏有着密切的关 系,热菜必热冷菜必涼才算合乎要求。我国以热食为主,走菜 时間和烹制时間的距离愈紧湊愈妙
。食品已經烹調之后最好馬 上食用,否則即影响其外表美说和滋美的味道。如魚热时味道 鮮美,冷涼之后即有脾味。还右許多帶音响的菜,如鍋巴尤魚
、 鍋巴蝦仁、鍋巴三鮮等不但要求同时制作,边制菜边炸鍋巴, 而F1要求以極为迅速的动作將菜上席,一手用碗端菜一手用盤 盛鍋巴,將菜傾于鍋巴上,湯汁与热汕和激即發出畢畢剝剝的 响声,所以称为雷菜。如上菜迟緩.鍋巴冷涼油性消失不酝而 疲,不再發岀响声,就失去其应有風味了。
