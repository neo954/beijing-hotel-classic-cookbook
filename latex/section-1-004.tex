4.      火候掌握

食品原料在經过初步加工之后,絕大部分(包括預备用来


制作冷菜的食品)都要进行热加工,因此热加工就成为整个烹 調过程的中心环节。热加工的关鍵在于能否灵活的掌握火候。 所謂火候是指火力强弱和操
作时間長短。火力一般有急火、烈 、火、猛火等,其威力無甚差別可統称为武火,又有緩火、細火、 弱火、微火等,可統称为文火。根据食品加工性質
不同,有的 使用武火,如爆炒菜,火力过弱,时間拖長則發疲。有的使用 文火,如煨燒菜,火急了則容易干或外熟內生。有的先用武火 后用文火,也有的先用文火后用或火。操作时除了需要注意 使用火力外,还要掌握加工的时間長短。有的食品愈煮愈嫩, 如腰子;有的煮过火則肉老發死,如魚。肉起
鍋退則其色由紅 变黑,蔬菜起鍋迟其維生素易走失。正确地掌握火力和操作时 間,不但要使作出的菜滋美适口,而且要尽可能地保持其原有 的营养成彷,以求有益于增进人体的健康。
