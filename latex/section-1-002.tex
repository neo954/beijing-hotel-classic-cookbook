2.      初步加工

食品的初步加工,即冷加工按其性質来說是热加工的准 备。这道工序对于烹制品的質量或佳或否有着很大的影响和作 用。因此要求愼重細心,不可了草
。加工时,燕需須用銀針或 鏡子將夾杂的細毛仔細地挑淨,魚翅去砂,海参去泥,如稍不 細致,就会使操作遭受困难,作出的菜就会苦澀难于入口,宰 割魚时要注意不把胆弄破,为了便于公鱗可用温开水燙之。炸 制的菜,如肉內有筋可將其剔出或用刀斬断,否則不隣。烹調' 上非常講究刀口,用刀割切的食品要力求刀口整齐均勻。刀口 好不但能够保証食品在热加工时受热均勻,不致彼生此熟或彼 熟此爛,而且可以增加菜的成色和美視。有許多菜不但
要求加 工細致,而且須进行复杂的美术工作:如蝴蝶海参耍佈置成婉' 如振翅欲飞之蝴蝶狀,釀龙头大蝦要点綴成騰云凌霧的龙狀, 以便以形象动人。
