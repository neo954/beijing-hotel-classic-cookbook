1.原料选擇

食品的烹調加工首先是由选擇原料开始的。質量优良的原 料是决定食品質量的前提条件。原料不好,即使厨师技术水平• 再高,加工条件再好、也断难作出好菜。

选擇原料时必須首先对食品进行感官檢査和物理檢查。L 官檢査是檢査食品的顏色、味道、包裝、商标等;物理檢査是 檢查食品的彈性、硬度和重量等。除此之外,在有条件的企業 里,还可以进行化学檢查和細菌檢査,通过这些檢查确定食品 新鮮与否,質量如何,有無腐敗变質和細菌占染等現象,不但 可以保証烹調加工的功效,而且可以保障消費者的安全,避免 因食物腐敗占染而引起的中毒和疾病笠不幸事故。

选擇原料时还必須將它們按照品种、部位和質量加以区 分。如魚有靑魚、桂魚、鯉魚、黃花魚、鯉魚、鰻魚之分,是 品种的不同;猪肉有腌尖、通脊、
五花、排骨、肘子、蹄爪肥. 膘之分,是部位的不同;鷄有雛母鷄、童鷄、筍鷄、油鷄、月 母鷄之分,是老嫩的不同。宜煎炸、宜熬煮、宜爆炒、宜蒸溜, 須按其質量优劣加以适当利用。合理地、正确地选擇和利用原 材料,把不同品种、不同部位、不同質量的原料按照不同用 途,充分地加以利用,不只对于烹調功效起着首要的作用,而 且对于节約原料也有着極为重要的意义。
