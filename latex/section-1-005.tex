5.調料应用

对于調味料和調味品的使用,各厨师均有不同,有的主味 重,有的則主味輕。下調料时必須首先考虑到它們的性質作用 和互相影响,並根据食品加工的
性質确定其数量标准。鹽味咸 糖味甜是人皆其知的,但各味調料加于一起所發生的互相作 用,則有待于进一步地探討和硏究,如使用料酒可以去腥避異 味,使用味精可以增加菜的香味和刺激食慾,皆有其科学道理, 各食品喜咸喜淡喜濃喜薄,有甚大差別,有的以淡为上,如淸 蒸諸菜,有的以咸为上,
多屬舊制。喜淡者咸則不鮮,喜咸者 淡則乏味。下調料时开始不宜过重,过堂欲輕即不易挽回。
